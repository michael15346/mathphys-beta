\section{Сопряжённое уравнение}
Рассмотрим задачу: дано линейное дифференциальное выражение
\[
	L[y] = a_n (x) y + a_{n-1} (x) y' + a_{n-2} (x) y'' + \dots + a_1 (x) y^{(n-1)} + a_0 (x) y^{(n)}, \tag{1}
\]
найти такую функцию $z(x)$, чтобы при умножении на неё (1) оно становилось точной производной по $x$ при любой $n$ раз дифференциремой функцией $y$. Эта функция $z(x)$ называется \textbf{множителем дифференциального выражения} $L[y]$. При этом мы будем предполагать, что функции $a_k(x), ~ k = \overline{0,n}$, непрерывные в рассматриваемом интервали и имеющие непрерывные производные всех порядков, которые войдут в наши формулы. Умножаем (1) на искомую функцию $z(x)$ и вычисляем
\[
	\int z(x) L[y] dx
\]
так, что каждый член интегрируем по частям, понижая порядок производной от $y$ до тех пор, пока под интегралом не останется множитель $y$. Таким образом, будем иметь:
\[
	\begin{aligned}
		\int a_n (x) yzdx = \int a_n (x) yzdx, \\
		\int a_{n-1} (x) y'zdx = a_{n-1} (x) zy - \int y(a_{n-1} (x)z)'dx \\
		\int a_{n-2} (x) y''zdx = a_{n-2}(x) zy' - \int (a_{n-2} (x)z)' y' = \\
		= a_{n-2} (x) zy' - (a_{n-2} (x) z)'y + \int y(a_{n-2}(x)z)''dx \\
		\dots \\
		\int a_1 (x) y^{(n-1)}zdx = a_1 (x)zy^{(n-2)} - (a_1(x)z)'y^{(n-3)}+(a_1(x)z)''y^{(n-4)} + \dots \\
		\dots + (-1)^{n-2} (a_1(x)z)^{(n-2)}y + (-1)^{n-1} \int y(a_1(x)z)^{(n-1)}dx \\
		\int a_0(x) y^{(n)}zdx = a_0(x)zy^{(n-1)} - (a_0(x)z)'y^{(n-2)} + (a_0(x)z)''y^{(n-3)} - \dots \\
		\dots + (-1)^{(n-1)} (a_0(x)z)^{(n-1)}y + (-1)^n \int y(a_0(x)z)^{(n)}dx
	\end{aligned}
\]
Собирая отдельно члены, не содержащие интегралов, и под общим знаком интеграла члены, содержащие квадратуру, получаем:
\[
	\begin{aligned}
		\int zL[y]dx = a_{n-1} (x) zy - (a_{n-2}z)'y + \dots + (-1)^{n-1} (a_0(x)z)^{(n-1)}y + \\
		+ a_{n-2}(x)zy'-(a_{n-3}(x)z)'y' + \dots + (-1)^{(n-2)}(a_0(x)z)^{(n-2)}y' + \dots \\
		\dots \\
		+ a_1(x)zy^{(n-2)} - (a_0(x)z)'y^{(n-2)} + a_0(x) zy^{(n-1)} + \\
		+ \int y\left[ a_n(x)z - (a_{n-1}(x)z)' + (a_{n-2}(x)z)''- \dots + (-1)^n (a_0(x)z)^{(n)}\right]dx
	\end{aligned}
\]
или, перенося интеграл в левую часть и вводя новые обозначения,
\[
	\int \left( zL[y] - yM[z]\right)dx = \Psi[y,z] \tag{2}
\]
Дифференциальное выражение
\[
	M[z] = a_n(x)z-(a_{n-1}(x)z)' + \dots + (-1)^{n-1} (a_1(x)z)^{(n-1)} + (-1)^n (a_0(x)z)^{(n)} \tag{3}
\]
называется \textbf{сопряжённым} с $L[y]$ дифференциальным выражением (или \textit{оператором}), а $\Psi[y,z]$ есть билинейная форма относительно $y,y',\dots,y^{(n-1)}$ с одной стороны и $z,z',\dots,z^{(n-1)}$ с другой стороны, а именно:
\[
	\begin{aligned}
		\Psi[y,z] = y\left(a_{n-1}(x)z-(a_{n-2}(x)z)'+\dots+(-1)^{n-1}(a_0(x)z)^{(n-1)}\right) + \\
		+ y'\left(a_{n-2}(x)z-(a_{n-3}(x)z)'+\dots + (-1)^{(n-2)}(a_0(x)z)^{(n-2)}\right)\\
		\dots \\
		+ y^{(n-2)}\left(a_1(x)z-(a_0(x)z)' \right) + y^{(n-1)} a_0(x)z
	\end{aligned} \tag{3'}
\]
Дифференциальное уравнение $n$-го порядка
\[
	M[z] = 0 \tag{4'}
\]
называется уравнением, сопряжённым с уравнением
\[
	L[y] = 0 \tag{4}
\]
Соотношение (2) есть тождество не только по $x_1$ - оно справедливо при любых функциях $y,z$. Если мы теперь возьмём в качестве $z$ решение уравнения (4'), $z=\overline{z}$, то (2) примет вид:
\[
	\int \overline{z} L[y]dx = \Psi\left[ y, \overline{z}\right].
\]
или, дифференцируя,
\[
	\overline{z} L[y] = \frac{d}{dx} \Psi \left[ y, \overline{z}\right]
\]
Таким образом, поставленная в начале задача решена: если умонжить данное дифференциальное выражение (1) на любое решение $\overline{z}$ сопряжённого уравнения (3'), то оно становится полной производной от дифференциального выражения $(n-1)$-го порядка $\Psi\left[ y, \overline{z}\right]$. Обратно, для того, чтобы функция $\overline{z}$ при умножении на $L[y]$ делала его такой производной при любой функции $y$, необходимо, чтобы $M\left[ \overline{z}\right]=0$.

В самом деле, если $\overline{z}$ есть какой-нибудь множитель выражения (1), то имеет место равенство:
\[
	\overline{z} L[y] = \frac{d}{dx} \Psi_1 [y] \tag{2'}
\]
где, как легко видеть, $Psi_1$ есть линейное выражение относительно $y, y', \dots, y^{(n-1)}$:
\[
	\Psi_1[y] = b_{n-1}(x) y^{(n-1)} + b_{n-2} (x) y^{(n-2)} + \dots + b_1 (x) y' + b_0 (x) y.
\]
С другой стороны, подставляя $\overline{z}$ вместо $z$ в (2) и дифференцируя по $x$, находим:
\[
	\overline{z} L[y] = \frac{d}{dx} \Psi\left[ y, \overline{z}\right] + y M\left[ \overline{z}\right] \tag{2''}
\]
Из (2'), (2'') получаем:
\[
	\frac{d}{dx} \left( \Psi_1[y] - \Psi\left[ y, \overline{z}\right]\right) - yM\left[ \overline{z}\right]= 0 \tag{2\textquotesingle \textquotesingle \textquotesingle}
\]
В левой части (2\textquotesingle \textquotesingle \textquotesingle) стоит линейное выражение $n$-го порядка относительно $y$; так как равенство нулю выполняется тождественно для любой функции $y$, то коэффициенты при $y$ и всех пго производных тождественно равны нулю, иначе (2'') было бы дифференциальным уравнением для $y$. Из вида (2') билинейного выражения для $\Psi$ следует:
\[
	\begin{aligned}
		b_{n-1} = a_0 (x) \overline{z}, b_{n-2} = a_1 (x) \overline{z} - \left(a_0(x) \overline{z} \right)', \dots, \\
		b_0 = a_{n-1} (x) \overline{z} - \left(a_{n-2} (x) \overline{z}\right)' + \dots + (-1)^{n-1} \left( a_0(x) \overline{z}\right)^{(n-1)}
	\end{aligned},
\]
т.е. $\Psi_1[y] = \Psi \left[ y, \overline{z}\right]$, и равенство (2\textquotesingle \textquotesingle \textquotesingle) даёт: $M\left[ \overline{z}\right]=0$.

Следовательно, мы можем сделать вывод:

Для того, чтобы функция $z(x)$ при всякой функции $y(x)$ обращала произведения $\overline{z}L[y]$ в точную производную, необходимо и достаточно, чтобы $\overline{z}$ являлось решением сопряжённого уравнения (4').

Каждое решение сопряжённого уравнения (4') является множителем уравнения (4'), при умножении на который левая часть (4) становится точной производной. Таким образом, (4) допускает первый интеграл:
\[
	\Psi\left[ y, \overline{z}\right] = C \tag{5}
\]
который сам является (неоднородным) уравнением порядка $n-1$. Очевидно, если нам дано уравнение $L[y]=f(x)$, то та же функция $\overline{z}$ является его множителем, и мы получим первый интеграл:
\[
	\Psi \left[ y, \overline{z}\right] = \int f(x) \overline{z}dx + C
\]
Если имеем линейное уравнение первого порядка
\[
	y'+P(x)y = Q(x),
\]
то уравнение, сопряжённое соответствующему однородному, будет
\[
	P(x)z-z'=0;
\]
его решение
\[
	\overline{z} = e^{\int P(x)dx}	
\]
будет множителем данного уравнения.

\textbf{Замечание 1}

Чтобы левая часть данного уравнения сама была точной производной, необходимо и достаточно, чтобы сопряжённое уравнение допускало решение $\overline{z}=1$, то есть чтобы коэффициент при $z$ в уравнении (4') был равен нулю. Раскрывая выражение (3) и подсчитывая в нём коэффициенты при $z'$, находим условие того, чтобы левая часть уравнения (4) была точной производной, в виде:
\[
	a_n (x) - \frac{d}{dx} a_{n-1} (x) + \frac{d^2}{dx^2} a_{n-2} (x) - \dots + (-1)^n \frac{d^n}{dx^n} a_0(x) = 0
\]
\textbf{Замечание 2}

Оператор $L[y]$ четного порядка $n=2m$ называется \textbf{самосопряжённым}: $L[y] \equiv M[y]$. Уравнение $L[y]=0$ называется в таком случае \textbf{самосопряжённым уравнением}. Для оператора второго порядка
\[
	L[y] = a_2 (x) y'' + a_1 (x) y' + a_0 (x) y
\]
сопряжённый оператор есть
\[
	\begin{aligned}
	M[z] = a_2 (x)z - (a_1(x)z)' + (a_0(x)z)'' =  \\
	= (a_2 (x) - a_1'(x) + a_0''(x))z+ (-a_1(x)+2a_0'(x))z'+a_0(x)z'';
\end{aligned}
\]
условия самосопряжённости
\[
	-a_1(x)+2a_0'(x)=a_1(x), a_2(x)-a_1'(x)+a_0''(x)=a_2(x)
\]
сводятся к одному первому: $a_1(x) = a_0'(x)$. Итак, самосопряжённый оператор второго порядка имеет вид:
\[
	(a_0(x)y')'+a_2(x)y.
\]
Рассмотрим линейное дифференциальное выражение второго порядка
\[
	Mu = \sum_{\alpha,\beta=1}^3 a_{\alpha\beta} \frac{\partial^2 u}{\partial x_\alpha\partial x_\beta} + \sum_{\alpha=1}^3 b_\alpha \frac{\partial u}{\partial x_\alpha} + cu, \tag{6}
\]
где $a_{\alpha\beta}, b_\alpha, c$ - некоторые функции $x_1, x_2, x_3$. Если функции $a_{\alpha\beta}$, а также функции
\[
	e_\alpha = b_\alpha - \sum_{\beta = 1}^3 \frac{\partial a_{\alpha \beta}}{\partial x_\beta} \tag{7}
\]
имеют непрерывные первые производные, то дифференциальному выражению $Mu$ можно придать вид
\[
	Mu = \sum_{\alpha,\beta=1}^3\frac{\partial}{\partial x_\beta} a_{\alpha \beta} \frac{\partial u}{\partial x_\alpha} + \sum_{\alpha=1}^3 e_\alpha \frac{\partial u}{\partial x_\alpha} + Cu \tag{8}
\]
Найдём функцию $v(x_1,x_2,x_3)$, при умножении на которую выражение $Mu$ выражение $vMu$ может быть представлено в виде суммы частных производных первого порядка по $x_1, x_2, x_3$.

\textit{остаток конспекта потерян в вечности...}
