\section{Общая схема метода Фурье}
Рассмотрим гиперболическое уравнение
\[
	\frac{\partial}{\partial x} \left( p(x) \frac{\partial u}{\partial x}\right) - q(x) u = \rho(x) \frac{\partial^2 u}{\partial t^2}, \tag{1}
\]
где $p(x), p'(x), q(x)$ и $\rho(x)$ - непрерывные функции при $0 \le x \le l$, причём $p(x) > 0, q(x) \ge 0, \rho(x) > 0$.

Пусть требуется найти решение уравнения (1), удовлетворяющее однородным граничным условиям
\[
	\begin{aligned}
		\alpha u(0,t) + \beta \frac{\partial u(0,t)}{\partial x} = 0, \\
		\gamma u(l,t) + \delta \frac{\partial u(l,t)}{\partial x} = 0,
	\end{aligned}
	\tag{2}
\]
где $\alpha, \beta, \gamma, \delta - const, \alpha^2 + \beta^2 \ne 0, \gamma^2 + \delta^2 \ne 0$, и начальным условиям
\[
	\left. u \right|_{t=0} = f(x), \left. \frac{\partial u}{\partial t}\right|_{t=0} = F(x) ~ (0 \le x \le l) \tag{3}
\]
Будем сначал искать нетривиальные решения (1) в виде произведения
\[
	u(x, t) = X(x) T(t), \tag{4}
\]
удовлетворяющего только граничным условиям (2).

Подставляя (4) в (1), получим
\[
	T(t) \frac{d}{dx} \left[ p(x) X'(x)\right] - q(x) X(x) T(t) = \rho(x) X(x) T''(t)
\]
или
\[
	\frac{\frac{d}{dx} \left[ p(x) X'(x)\right] - q(x) X(x)}{\rho(x) X(x)} = \frac{T''(t)}{T(t)} \tag{5}
\]
Левая часть (5) зависит только от $x$, правая - только от $t$, значит $(5) = -\lambda = const$. Тогда из (5) получим два дифференциальных уравнения:
\[
	T''(t) + \lambda T(t) = 0, \tag{6}
\]
\[
	\frac{d}{dx} \left[ p(x) X'(x) \right] + \left[ \lambda \rho(x) - q(x) \right] X(x) = 0 \tag{7}
\]
Чтобы получить нетривиальные решения уравнения (1) вида (4), удовлетворяющие граничным условиям (2), необходимо, чтобы $X(x)$ удовлетворяла граничным условиям
\[
	\begin{aligned}
		\alpha X(0) + \beta X'(0) = 0, \\
		\gamma X(l) + \delta X'(l) = 0 
	\end{aligned} \tag{8}
\]
Таким образом, приходим к следующей задаче \textbf{Штурма-Лиувилля} о собвственных числах: найти такие значения $\lambda$, при которых существуют нетривиальные решения (7), удовлетворяющие граничным условиям (8).

Эта задача не при всяком $\lambda$ имеет отличные от нулевого решения. Те значения $\lambda$, при которых задача (7)-(8) имеет нетривиальные решения, называются \textbf{собственными числами}, а сами эти решения - \textbf{собственными функциями}, соответствующими этому с.ч. $\lambda$. (собственные функции собственных чисел = с.ф. с.ч.) В силу однородности (7)-(8), собственные функции определяются с точностью до постоянного множителя. Выберем его так, чтобы
\[
	\int\limits_0^l \rho(x) X^2(x) dx = 1 \tag{9}
\]
Собственные функции, удовлетворяющие (9), будем называть \textbf{нормированными}.

Установим некоторые общие свойства с.ф. с.ч. задачи Штурма-Лиувилля:
\begin{enumerate}
	\item{Всякому с.ч. соответствует только одна линейно независимая с.ф.

			Действительно, предположим, что при некотором $\lambda$ существуют два линейно независимых решения (7), удовлетворяющих (8). Тогда оказалось бы, что и общее решение уравнения (7) удовлетворяет (8). Но этого быть не может, так как всегда можно найти решение уравнения (7) при таких начальных данных $X(0), X'(0)$, например, $X(0)=\alpha, X'(0)=\beta$, которые не удовлетворяют первому из граничных условий (8).
		}
		\item{Собственные функции, соответствующие различным с.ч., ортогональны с весом $\rho(x)$, то есть
				\[
					\int\limits_0^l \rho(x) X_1 (x) X_2 (x) dx = 0 \tag{10}
				\]
			Пусть $\lambda_1, \lambda_2$ - два различных с.ч., а $X_1(x), X_2(x)$ - соответствующие им с.ф., так что
			\[
				\begin{aligned}
					\frac{d}{dx} \left[ p(x) X_1'(x)\right] + \left[ \lambda_1 \rho(x) - q(x)\right]X_1(x) = 0 \\
					\frac{d}{dx} \left[ p(x) X_2'(x)\right] + \left[ \lambda_2 \rho(x) - q(x)\right]X_2 (x) = 0
				\end{aligned}
			\]
			Умножим первое равенство на $X_2(x)$, второе - на $X_1(x)$ и вычтем одно из другого почленно, получаем равенство
			\[
				X_2(x) \frac{d}{dx} \left[ p(x) X_1'(x)\right] - X_1 (x) \frac{d}{dx} \left[ X_2'(x) p(x)\right] + (\lambda_1 - \lambda_2) \rho(x) X_1 (x) X_2 (x) = 0,
			\]
			которое можно переписать в виде
			\[
				(\lambda_1 - \lambda_2) \rho(x) X_1 (x) X_2 (x) + \frac{d}{dx} \left\{ p(x) \left[ X_2(x) X_1'(x) - X_1 (x) X_2' (x)\right]\right\} = 0
			\]
			Интегрируя это равенство по $x$ от $0$ до $l$, получим
			\[
				(\lambda_2 - \lambda_1) \int\limits_0^l \rho(x) X_2 (x) X_1 (x) dx = p(x) \left[ X_2 (x) X_1' (x) - X_1 (x) X_2' (x)\right]_{x=0}^{x=l}
			\]
			Приняв во внимание граничные условия (8), легко убеждаемся, что
			\[
				(\lambda_2 - \lambda_1) \int\limits_0^l \rho(x) X_1 (x) X_2 (x) dx = 0,
			\]
			откуда в силу $\lambda_1 \ne \lambda_2$,
			\[
				\int\limits_0^l \rho(x) X_1 (x) X_2 (x) dx = 0,
			\]
			$\square$.
			}
			\item{Все собственные числа вещественны.

					В самом деле, допустим, что существует с.ч. $\lambda \in \mathbb{C}$, которому соответствует с.ф. $X(x)$. Тогда $\overline{\lambda}$ также будет с.ч., а $\overline{X}(x)$ - с.ф., так как коэффициенты в (7) и (8) вещественны. Из условий ортогональности
					\[
						\int\limits_0^l \rho(x) X(x) \overline{X}(x) dx = \int\limits_0^l \rho(x) \left| X(x)\right|^2 dx=0,
					\]
					следует, что $X(x) \equiv 0$, т.е. $\lambda$ не является собственным.
				}
			\item{Существует бесконечное множество вещественных с.ч.
					\[
						\lambda_1 < \lambda_2 < \lambda_3 < \dots < \lambda_n \dots, ~ \lim_{n \to \infty} \lambda_n = +\infty
					\]	
				Прежде чем перейти к обоснованию этого утверждениия, укажем ещё одно важное свойство с.ч.. Пусть $\lambda_k$ - с.ч., а $X_k(x)$ - с.ф., образующие ортогональную нормированную систему. Имеем
				\[
					\frac{d}{dx} \left[ p(x) X_k'(x)\right] - q(x) X_k (x) = -\lambda_k \rho(x) X_k (x)
				\]
				Умножая обе части на $X_k (x)$, интегрируя и принимая во внимание (9), получаем
				\[
					\lambda_k = -\int\limits_0^l \left\{ \frac{d}{dx} \left[ p(x) X_k'(x)\right]-q(x) X_k(x)\right\} X_k(x) dx,
				\]
				откуда, интегрируя первое млагаемое по частям, придем к формуле:
				\[
					\lambda_k = \int\limits_0^l \left[ p(x) {X_k'}^2+q(x) X_k^2(x)\right]dx - \left[p(x) X_k(x) X_k'(x) \right]_{x=0}^{x=l} \tag{11}
				\]
				Допустим, что $p(x)>0, q(x)\ge 0, \rho(x)>0$ и, кроме того,
				\[
					\left[ p(x) X_k(x) X_k'(x)\right]_{x=0}^{x=l} \le 0 \tag{11a}
				\]
				Тогда из (9) следует, что все с.ч. задачи (7)-(8) неотрицательны.

				Условие (11a) выполняется при наиболее часто встречающихся в приложениях граничных условиях:
				\[
					X(0)=0, X(l)=0 \tag{8a},
				\]
				\[
					X'(0)-h_1X(0)=0, X'(l) + h_2 X(l) = 0, ~ h_1 \ge 0, h_2 \ge 0 \tag{8b}
				\]
				В заключение отметим, что с.ф. $X_k(x)$ граничной задачи (7)-(8a) или (7)-(8b) (если $h_1 = h_2 = 0$, то $q(x) \ge q_0 > 0$) образуют полную систему.
					}
\subsection{Определение 1}
Система функций $\varphi_1(x), \varphi_2(x), \dots, \varphi_n (x), \dots$ называется \textbf{полной}, если не существует отличной от тождественно равной нулю, суммируемой с квадратом функции, ортогональной ко всем функциям системы.

Обратимся теперь к (6). Его общее решение при $\lambda = \lambda_k$, которое обозначим $T_k(t)$, имеет вид
\[
	T_k (t) = A_k \cos \sqrt{\lambda_k} t + B_k \sin \sqrt{\lambda_k} t,
\]
где $A_k, B_k$ - произвольные постоянные.

Каждая функция
\[
	u_k(x,t) = X_k(x) T_k(t) = \left( A_k \cos \sqrt{\lambda_k} t + B_k \sin \sqrt{\lambda_k} t\right)X_k (x)
\]
будем решением уравнения (1), удовлетворяющим граничным условиям (2).

Чтобы удовлетворить начальным условиям (3), составим ряд
\[
	u(x,t) = \sum_{k=1}^\infty \left( A_k \cos \sqrt{\lambda_k} t + B_k \sin \sqrt{\lambda_k} t\right) X_k (x)\tag{12}
\]
Если этот ряд сходится равномерно вместе с рядами, получаемыми из него двухкратным почленным дифференцированием, то сумма его, очевидно, будет решением (1), удовлетворяющим граничным условиям (2). Для выполнения начальных условий (3) необходимо, чтобы
\[
	\left. u\right|_{t=0} = f(x) = \sum_{k=1}^\infty A_k X_k (x), \tag{13}
\]
\[
	\left. \frac{\partial u}{\partial t}\right|_{t=0} = F(x) = \sum_{k=1}^\infty B_k \sqrt{\lambda_k} X_k (x) \tag{14}
\]
Таким образом, мы пришли к задаче о разложении произвольной функции в ряд по с.ф. $X_k(x)$ граничной задачи (7)-(8).

Предполагая, что (13) и (14) сходятся равномерно, можно определить $A_k, B_k$, умножить обе части равенств (13)-(14) на $\rho(x) X_k(x)$ и проинтегрировать по $x$ от $0$ до $l$. Тогда, принимая во внимание (9)-(10, получим
\[
	A_k = \int\limits_0^l \rho(x) f(x) X_k (x) dx, ~ B_k = \frac{1}{\sqrt{\lambda_k}} \int\limits_0^l \rho(x) F(x) X_k(x)dx
\]
Подставив эти значения $A_k, B_k$ в (12), получим решение смешанной задачи (1)-(3), если ряд (12) и ряды, полученные из него двухкратным почленным дифференцированием по $x$ и $t$, равномерно сходятся.

\textbf{Замечание}

Метод Фурье применим и в случае многих пространственных переменных для гиперболических уравнений специального вида, а также для уравнений эллиптического и параболического типов.

Рассмотрим малые колебания однородной прямоугольной мембраны со сторонами $p, q$, закреплённой по контуру.

Как известно, эта задача сводится к решению волнового уравнения
\[
	\frac{\partial^2 u}{\partial t^2} = a^2 \left( \frac{\partial^2 u}{\partial x^2} + \frac{\partial^2 u}{\partial y^2}\right) \tag{15}
\]
при граничных условиях
\[
	\begin{aligned}
		\left. u\right|_{x=0} = 0, ~ \left. u\right|_{x=p} = 0 \\
				\left. u\right|_{y=0} = 0, ~ \left. u\right|_{y=q} = 0
					\end{aligned} \tag{16}
\]
и начальных условиях
\[
	\left. u\right|_{t=0} = f(x,y), \left. \frac{\partial u}{\partial t}\right|_{t=0} = F(x,y) \tag{17}
\]
Будем искать частные решения (15) в виде
\[
	u(x,y,t) = T(t) v(x,y), \tag{18}
\]
удовлетворяющие граничным условиям (16).

Подставив (18) в (15), получим
\[
	\frac{T''(t)}{a^2 T(t)} = \frac{v_{xx} + v_{yy}}{v}
\]
Очевидно, что это равенство может иметь место, только когда обе его части равны одной и той же постоянной величине. Обозначим эту постоянную через $-k^2$, и, принимая во внимание (16), найдём, что
\[
	T''(t) + (ak)^2 T(t) = 0 \tag{19}
\]
\[
	v_{xx} + v_{yy} + k^2 v = 0 \tag{20}
\]
\[
	\begin{aligned}
	\left. v\right|_{x=0} = 0, \left. v\right|_{x=p} = 0 \\
			\left. v\right|_{y=0} = 0, \left. v\right|_{y=q} = 0
				\end{aligned} \tag{21}
\]
Граничную задачу (20)-(21) будем решать методом Фурье, полагая
\[
	v(x,y) = X(x) Y(y) \tag{22}
\]
Подставляя (22) в (20), получим
\[
	\frac{Y''(y)}{Y(y)} + k^2 = -\frac{X''(x)}{X(x)}
\]
откуда получаем два уравнения:
\[
	X''(x) + k_1^2 X(x) = 0, Y''(y) + k_2^2 Y(y) = 0 \tag{23}
\]
где
\[
	k^2 = k_1^2 + k_2^2 \tag{24}
\]
Общие решения (23), как известно, имеют следующий вид:
\[
	\begin{aligned}
		X(x) = C_1 \cos k_1 x + C_2 \sin k_1 x \\
		Y(y) = C_3 \cos k_2 y + C_4 \sin k_2 y
	\end{aligned} \tag{25}
\]
Из граничных условия (21) получим
\[
	\begin{aligned}
		X(0) = 0, X(p) = 0 \\
		Y(0) = 0, Y(q) = 0
	\end{aligned} \tag{26}
\]
откуда ясно, что $C_1 = C_3 = 0$, и если мы положим $C_2 = C_4 = 1$, то окажется:
\[
	X(x) = \sin k_1 x, ~ Y(y) = \sin k_2 y \tag{27}
\]
причём должно быть
\[
	\sin k_1 p = 0, \sin k_2 q = 0 \tag{28}
\]
Из уравнения (28) вытекает, что $k_2$ и $k_1$ имеют бесконечное множество значений:
\[
	k_{1m} = \frac{m\pi}{p}, k_{2n} = \frac{n \pi}{q} (m,n=1,2,3,\dots)
\]
Тогда из равенства (24) получим соответствующие значения $k^2$:
\[
	k_{mn}^2 = k_{1m}^2 + k_{2n}^2 = \pi^2 \left( \frac{m^2}{p^2} + \frac{n^2}{q^2}\right) \tag{29}
\]
Таким образом, с.ч. (29) соответствуют с.ф.
\[
	v_{mn} = \sin \frac{m \pi x}{p} \sin \frac{n \pi y}{q} \tag{30}
\]
граничной задачи (20)-(21).

Обращясь теперь к (19), видим, что для каждого с.ч. $k^2=k_{mn}^2$ его общее решение имеет вид
\[
	T_{mn} (t) = A_{mn} \cos a k_{mn} t + B_{mn} \sin a k_{mn} t \tag{31}
\]
Таким образом, в силу (18), (30) и (31), частные решения уравнения (15), удовлетворяющие граничным условиям (16), имеют вид:
\[
	u_{mn} (x,y,t) = \left( A_{mn} \cos a k_{mn} t + B_{mn} \sin a k_{mn} t\right) \sin \frac{m \pi x}{p} \sin \frac{n \pi y}{q} \tag{32}
\]
Чтобы удовлетворить начальным условиям (17), составим ряд
\[
	u(x,y,t) = \sum_{m=1}^\infty \sum_{n=1}^\infty \left( A_{mn} \cos a k_{mn} t + B_{mn} \sin a k_{mn} t\right) \sin \frac{m \pi x}{p} \sin \frac{n \pi y}{q} \tag{33}
\]
Если этот ряд равномерно сходится, так же как и ряды, полученные из него двукратным почленным дифференцированием по $x, y$ и $t$, то сумма его, очевидно, будет удовлетворять (15) и (16). Для выполнения начальных условий (17) необходимо, чтобы
\[
	\left. u\right|_{t=0} = f(x,y) = \sum_{m=1}^\infty \sum_{n=1}^\infty A_{mn} \sin \frac{m \pi x}{p} \sin \frac{n \pi y}{q}, \tag{34}
\]
\[
	\left. \frac{\partial u}{\partial t}\right|_{t=0} = F(x,y) = \sum_{m=1}^\infty \sum_{n=1}^\infty a k_{mn} B_{mn} \sin \frac{m \pi x}{p} \sin \frac{n \pi y}{q} \tag{35}
\]
Предполагая, что ряды (34), (35) сходятся равномерно, мы можем определить $A_{mn}, B_{mn}$, умножив обе части (34) и (35) на
\[
	\sin \frac{m_1 \pi x}{p} \sin \frac{n_1 \pi y}{q}
\]
и проинтегрировав по $x$ от $0$ до $p$ и по $y$ от $0$ до $q$. Тогда, приняв во внимание, что
\[
	\begin{aligned}
	\int\limits_0^p \int\limits_0^q \sin \frac{m \pi x}{p} \sin \frac{n \pi y}{q} \sin \frac{m_1 \pi x}{p} \sin \frac{n_1 \pi y}{q} dxdy = \\
	= \left\{ \begin{aligned} 0, ~& \text{ если } m\ne m_1 \text{ или } n\ne n_1 \\
	\frac{pq}{4}, ~& \text{ если } m=m_1 \text{ и } n=n_1\end{aligned}\right.
\end{aligned}
\]
получаем
\[
	\begin{aligned}
	A_{mn} = \frac{4}{pq} \int\limits_0^p \int\limits_0^q f(x,y) \sin \frac{m \pi x}{p} \sin \frac{n \pi y}{q} dxdy, \\
	B_{mn} = \frac{4}{a p q k_{mn}} \int\limits_0^p \int\limits_0^q F(x,y) \sin \frac{m \pi x}{p} \sin \frac{n \pi y}{q} dxdy 
\end{aligned} \tag{36}
\]
Решение (33) можно записать также в виде
\[
	u(x,y,t) = \sum_{m=1}^\infty \sum_{n=1}^\infty M_{mn} \sin \frac{m \pi x}{p} \sin \frac{n \pi y}{q} \sin \left( a k_{mn} t + \varphi_{mn}\right) \tag{37}
\]
где
\[
	M_{mn} = \sqrt{A_{mn}^2 + B_{mn}^2}, ~ \varphi_{mn} = \arctg \frac{A_{mn}}{B_{mn}}
\]
\end{enumerate}
