\section{Примеры из физики}
Обозначения: $\frac{\partial^2 u}{\partial t^2} = u_{tt}, \Delta = \partial_x^2 + \partial_y^2 + \partial_z^2$
\begin{enumerate}
	\item{Волновые уравнения.
			\[
				u_{tt} = a^2 u_{xx}
		\] - уравнение малых колебаний струны
		\[
				u_{tt} = a^2 u_{xx} + g
		\] - уравнение малых колебаний струны с учетом силы тяжести, где $g$ - потенциал массовой силы.
		\[
			u_{tt} = a^2 \left( u_{xx} + u_{yy}\right) + g
		\] - уравнение мембраны
		\[
			u_{tt} = a^2 \left( u_{xx} + u_{yy} + u_{zz}\right) + g
		\] - уравнение акустики
	}
\item{\[
			u_t = a^2 u_{xx} + g
	\] - уравнение теплопроводности для стержня.
	\[
		u_t = a^2 \left( u_{xx} + u_{yy}\right) + g
	\] - уравнение теплопроводности для пластины.
	\[
		u_t = a^2 \left( u_{xx} + u_{yy} + u_{zz}\right) + g
	\] - уравнение теплопроводности для тела.
}
\item{
		Уравнение Лапласа:
		\[
			\Delta u = 0
		\]
		Уравнение Пуассона:
		\[
			\Delta u = -g
		\]
	}
\item{
		Уравнение вибраций для стержня:
		\[
			u_{tt} + a^2 u_{xxxx} = g
		\]
		Для пластины:
		\[
			u_{tt} + a^2 \left(u_{xxxx} + 2 u_{xxyy} + u_{yyyy}\right) = g
		\]
	}
\item{
		Уравнение Шрёдингера (квантовая механика):
		\[
			i \hslash u_t = Hu
		\]
		где $i$ - мнимая единица, $\hslash$ - постоянная Планка, $H$ - оператор Гамильтона. Для одномерной частицы получится:
		\[
			H = E_\text{К} + E_\text{П}
		\]
		В классической механике это будет:
\[
	E_\text{К} = \frac{mv^2}{2} = \frac{m\left| \left| \overline{v}\right| \right|^2}{2} = \frac{\left| \left| \overline{P}\right| \right|^2}{2m}
\]
В квантовой механике:
\[
	\begin{aligned}
		\overline{P} = -i \hslash \nabla \\
		E_\text{К} = -\frac{\hslash^2}{2m} \Delta
	\end{aligned}
\]
То есть $E_\text{К}$ - это оператор. $E_\text{П}$ - потенциал внешнего поля, $E_\text{П}u = U(x) \cdot u$. Получили уравнение:
\[
	i \hslash u_t = -\frac{\hslash^2}{2m} u_{xx} + U(x) \cdot u
\] - нерелятивистская одномерная частица.
	}
	\begin{comment}
	\item{
			Уравнение ОТО (общая теория относительности)
			\[
				\begin{aligned}
					g_{vm} = \begin{pmatrix} c^2 & 0 & 0 & 0 \\ 0 & -1 & 0 & 0 \\ 0 & 0 & -1 & 0 \\ 0 & 0 & 0 & -1\end{pmatrix}	
				\end{aligned}
			\]
			Вот это метрический тензор.
		}
	\end{comment}
	\item{
			Уравнение Навье--Стокса. Уравнение для вязкой несжимаемой жидкости:
			\[
				\begin{aligned}
				\overline{w}_t = \underbrace{-\left( \overline{w}, \nabla\right) \overline{w}}_{\text{перенос жидкости}} + \underbrace{a^2 \Delta \overline{w}}_\text{вязкость} - \frac{1}{\underbrace{\rho}_{\text{плотность}}} \nabla \underbrace{P}_{\text{давление}} + \underbrace{g}_{\text{потенциальная массовая сила}} \\
				\mathrm{div} ~ \overline{w} = 0
			\end{aligned}
			\]
			при этом
			\[
				\begin{aligned}
					P = P(t,x,y,z) \in \mathbb{R}^1 \\
					\overline{w} = \overline{w} (t,x,y,z) \in \mathbb{R}^3
				\end{aligned}
			\]
		}
\end{enumerate}
